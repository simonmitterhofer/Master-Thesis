\section*{Abstract} \label{sec:Abstract}

This thesis investigates the Hypergeometric Volatility Model as a flexible framework for option pricing under rough and persistent volatility dynamics. Motivated by empirical evidence of volatility clustering, long memory, and rough sample paths, the model employs a hypergeometric mixing measure to define a latent signal process that drives stochastic volatility. Finite-dimensional approximations enable efficient simulation and practical implementation.\\
A comprehensive simulation study is conducted to evaluate the model's ability to reproduce key stylized facts of implied volatility surfaces. Five quality criteria are defined, including put-call parity, smile convexity, negative and increasing at-the-money skew, and approximate power-law decay. More than 18{,}000 parameter constellations are simulated, and the analysis identifies robust parameter regions, particularly roughness indices between $0.10$ and $0.25$, correlation parameters in $[-0.6, -0.2]$, and moderate volatility scales below $1.4$, that consistently generate realistic outcomes.\\
The model is further validated empirically through calibration to S\&P~500 option data. The calibrated surfaces successfully reproduce essential market features, although the best-fit parameters partially deviate from those found optimal in the simulation study. This highlights the trade-off between reproducing stylized facts and achieving close market fit.\\
Overall, the results demonstrate that the Hypergeometric Volatility Model offers a tractable and empirically consistent framework for modeling option-implied volatility, while also pointing to open challenges in calibration methodology and model robustness.