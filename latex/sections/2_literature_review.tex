\section{Literature Review} \label{sec:LiteratureReview}

This section reviews key developments in volatility modeling, starting with classical models and their limitations, followed by empirical findings on volatility roughness and persistence. It concludes with modern modeling approaches and practical methods for simulation and approximation.


\subsection{Empirical Properties of Volatility and Limitations of Classical Models} \label{subsec:EmpiricalVolatility}

Classical models of asset price dynamics typically treat volatility as either a deterministic function of time or a continuous stochastic process driven by standard Brownian motion. The Black-Scholes model \citep{BlackScholes1973} assumes that volatility is constant, while the Heston model \citep{Heston1993} introduces stochastic volatility using a mean-reverting square-root process.

Although these models are widely used and mathematically convenient, they do not fully capture how volatility behaves in real financial markets. Volatility tends to cluster, meaning that high or low volatility often persists over time. Realized volatility also exhibits long memory, i.e., past values influence future values over extended periods. These are part of the so-called “stylized facts” of financial time series \citep{Cont2001}.

In the options market, implied volatility surfaces show patterns that classical models fail to reproduce. These include volatility smiles and skews \citep{Rubinstein1985}, as well as a power-law decay of the at-the-money (ATM) volatility skew with increasing time to maturity \citep{GatheralJaissonRosenbaum2018}. As a result, option prices from these models often deviate significantly from observed market prices, especially for short maturities or deep out-of-the-money options.


\subsection{Motivation for Rough and Persistent Volatility Models} \label{subsec:MotivationRoughPersistent}

The limitations of classical models have motivated researchers to seek more realistic representations of volatility. Recent empirical evidence indicates that volatility is not only stochastic, but exhibits two key properties that standard models fail to capture: roughness and persistence.

\citet{GatheralJaissonRosenbaum2018} find that realized log-volatility behaves like a fractional Brownian motion with Hurst parameter $H \approx 0.1$. This low Hurst parameter implies highly irregular, or rough, sample paths with negatively correlated increments. This behavior cannot be captured by models driven by standard Brownian motion.

At the same time, volatility exhibits long memory, meaning that its autocorrelations decay slowly over time. This behavior has been documented in \citet{ComteRenault1998} and further studied in \citet{BennedsenLundePakkanen2021}, who develop kernel-based approaches to model long memory.

Further theoretical support comes from \citet{Fukasawa2017}, who shows that power-law behavior in the ATM skew can arise under weak regularity conditions on the underlying volatility process. This result suggests that the observed patterns reflect fundamental structural market properties rather than statistical artifacts.

These empirical and theoretical findings have motivated the development of new volatility models that aim to capture both short-term irregularity and long-term memory. Unlike traditional Markovian approaches, these models explicitly incorporate roughness and persistence, leading to improved performance in derivatives pricing and risk management applications.


\subsection{Modern Volatility Models} \label{subsec:ModernVolModels}

To incorporate roughness and long memory into volatility modeling, several new models have been proposed. These approaches differ in how they capture the time-series properties of volatility and in the extent to which they reproduce key features of implied volatility surfaces.

\citet{GatheralJaissonRosenbaum2018} introduce the Rough Fractional Stochastic Volatility (RFSV) model, where log-volatility follows a fractional Ornstein-Uhlenbeck process driven by fractional Brownian motion with Hurst parameter $H < 0.5$. This specification generates rough sample paths with long-memory characteristics. In option markets, it reproduces the empirically observed power-law decay of the at-the-money skew.

The Rough Heston model by \citet{ElEuchRosenbaum2019} extends the classical Heston framework by incorporating a fractional kernel in the volatility dynamics. This introduces roughness into the variance process. Although the model is no longer Markovian, it still allows for semi-closed-form characteristic functions, making it suitable for European option pricing.

Other generalizations include affine Volterra processes and affine rough volatility models, such as those in \citet{AbiJaberLarssonPulido2019}. Their contribution lies in flexibly capturing memory effects through stochastic convolution equations, while for option pricing they retain affine structures that facilitate tractable computations.

\citet{BennedsenLundePakkanen2021} propose the Brownian semistationary (BSS) framework, which specifies volatility as a convolution integral of Brownian motion with deterministic kernels. This approach provides a flexible structure for modeling the time-series properties of volatility, as the kernel can be adjusted to capture both roughness and persistence.

Extending the convolution-based approach, \citet{Damian2021} introduces the Hypergeometric Volatility Model, which employs a hypergeometric mixing measure to define the convolution kernel. This structure gives explicit control over both short- and long-term dependencies in volatility dynamics. From an option-pricing perspective, the model can generate implied volatility surfaces with realistic skews and smile convexity. The hypergeometric framework serves as the basis for the analysis in this thesis and is presented in detail in Section~\ref{sec:ModelingFramework}.


\subsection{Simulation and Approximation Methods} \label{subsec:SimulationApproximation}

Rough volatility models often involve infinite-dimensional processes that cannot be simulated exactly, making approximation schemes necessary for practical implementation.

A common approach relies on approximating these processes through mixtures of Ornstein-Uhlenbeck (OU) processes. \citet{CarmonaCoutin1998} show that stochastic convolutions with Laplace-type kernels can be expressed as infinite mixtures of OU processes with varying mean-reversion speeds. This decomposition preserves key properties such as stationarity and mean reversion.

\citet{CarmonaCoutinMontseny2000} propose a finite-dimensional approximation by discretizing the integral over the mixing measure. The result is a weighted sum of OU processes that can be simulated efficiently.

For this spatial discretization, \citet{DamianFrey2024} suggest the use of a geometric partitioning of the integration domain, ensuring that each of the intervals has the same mass under the mixing measure. This discretization method is implemented in this thesis for simulating the latent signal process and computing asset price dynamics under the Hypergeometric Volatility Model.