\section{Introduction} \label{sec:Introduction}

The modeling of volatility is a central challenge in modern finance. A good understanding and realistic description of volatility dynamics is essential for accurate derivatives pricing and applications in risk management. A common reference is the seminal work by \cite{BlackScholes1973}, which remains one of the most widely used mathematical finance papers in practice. In their framework, the asset price is modeled with constant volatility, which allows for an elegant closed-form solution for European option prices. While this approach is mathematically tractable, it does not reflect the dynamics observed in real markets, where volatility is stochastic and exhibits complex time-varying patterns.

Classical extensions of the Black-Scholes model, such as the Heston model \citep{Heston1993}, account for stochastic volatility and provide greater flexibility. Nevertheless, these models still fail to reproduce important empirical features observed in financial markets. At the time-series level, realized volatility exhibits clustering, long memory, and roughness. Clustering refers to the persistence of high or low volatility periods, while long memory is characterized by slowly decaying autocorrelations over extended time horizons. Roughness, as documented by \cite{GatheralJaissonRosenbaum2018}, means that log-volatility follows sample paths that are highly irregular, resembling fractional Brownian motion with a Hurst parameter significantly below $0.5$.

At the same time, option market data reveal characteristic patterns in implied volatility surfaces that classical models cannot capture. These include volatility smiles and skews \citep{Rubinstein1985}, as well as a power-law decay in the at-the-money skew with increasing time to maturity \citep{GatheralJaissonRosenbaum2018}. Since such patterns are structural features of option markets, models that fail to reproduce them systematically misprice options, particularly at short maturities and for deep in- or out-of-the-money strikes.

These empirical shortcomings have motivated the development of rough volatility models, which incorporate highly irregular sample paths, and persistent volatility models, which capture slowly decaying autocorrelations. Recent approaches, such as the Rough Heston model \citep{ElEuchRosenbaum2019} and Brownian semistationary frameworks \citep{BennedsenLundePakkanen2021}, have demonstrated that explicitly accounting for roughness and long memory significantly improves consistency with observed market behavior.

Building on this line of research, \cite{Damian2021} introduced the Hypergeometric Volatility Model. This model employs a hypergeometric mixing measure to define a latent signal process that drives volatility dynamics. It provides explicit control over both roughness and persistence parameters, offering a unified framework that can capture the main stylized facts of financial markets. Importantly, the model admits computationally tractable finite-dimensional approximations that enable efficient simulation and practical implementation.

This thesis contributes to the literature by conducting a comprehensive simulation-based study of the Hypergeometric Volatility Model. The analysis focuses on the model's ability to reproduce stylized facts of implied volatility surfaces, systematically identifies parameter regions that lead to stable and realistic outcomes, and evaluates the model's empirical performance through calibration to S\&P~500 option data. In doing so, the thesis complements the theoretical foundations of the model with extensive numerical analysis and provides practical insights for its implementation.

The structure of the thesis is as follows. Section~\ref{sec:LiteratureReview} reviews the literature on volatility stylized facts, limitations of classical models, recent modeling approaches, and simulation techniques. Section~\ref{sec:ModelingFramework} introduces the Hypergeometric Volatility Model and describes its mathematical framework. Section~\ref{sec:SimulationMethodology} presents the simulation methodology for generating option prices and implied volatility surfaces. Section~\ref{sec:SimulationResultsAnalysis} reports comprehensive simulation results and analyzes the influence of model parameters. Section~\ref{sec:EmpiricalValidation} provides empirical validation through calibration to S\&P~500 option data. Finally, Section~\ref{sec:Conclusion} summarizes the findings and discusses their implications for the practical application of rough volatility models.