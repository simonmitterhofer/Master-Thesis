\section{Conclusion} \label{sec:Conclusion}

This thesis investigated the Hypergeometric Volatility Model as a framework for capturing key features of implied volatility surfaces observed in financial markets. A comprehensive simulation study, complemented by empirical validation using S\&P~500 option data, was conducted to evaluate the model's ability to reproduce empirical stylized facts and to identify parameter configurations that produce realistic outcomes.

To assess how realistic simulated surfaces are, five quality criteria were established: put-call parity, smile convexity, negative at-the-money (ATM) skew, increasing skew term structure, and approximate power-law decay. Through the systematic analysis of over 18{,}000 parameter combinations, the influence of roughness, persistence, correlation, and volatility scaling parameters on surface quality was examined. The results identified several key parameter ranges for optimal performance: roughness indices $H$ between $0.10$ and $0.25$, correlation parameters $\rho$ in the range $[-0.6, -0.2]$, and volatility scale values $a$ below $1.4$. While these parameters showed the strongest effects, careful selection of base volatility and persistence parameters remained important for achieving realistic dynamics.

The overall success rate across all parameter configurations was $54\%$. Individual criteria showed varying performance levels, with put-call parity satisfied in $93\%$ of cases and negative skew in $99\%$ of cases, while smile convexity proved more challenging at $66\%$. This analysis provides practical guidance for model implementation by identifying parameter regions that consistently produce market-realistic outcomes.
