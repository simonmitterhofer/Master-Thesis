\section{Conclusion} \label{sec:Conclusion}

This thesis examined the Hypergeometric Volatility Model as a framework for reproducing key features of implied volatility surfaces observed in financial markets. A broad simulation study, combined with empirical validation using S\&P~500 option data, was conducted to evaluate the model's ability to match empirical stylized facts and to determine parameter settings that generate realistic surfaces.

To assess how realistic simulated surfaces are, five quality criteria were established: put-call parity, smile convexity, negative at-the-money (ATM) skew, increasing skew term structure, and approximate power-law decay. More than 18{,}000 parameter combinations were simulated to analyze the effect of roughness, persistence, correlation, and volatility scaling on the implied volatility surface. The results identified key parameter ranges for optimal model performance: roughness indices $H$ between $0.10$ and $0.25$, correlation parameters $\rho$ in the range $[-0.6, -0.2]$, and volatility scale values $a$ below $1.4$. The persistence parameter $\gamma_2$ showed relatively modest effects, although slightly better performance was observed for smaller values. While roughness, correlation, and volatility scale showed the strongest effects, the choice of base volatility also contributed to surface realism.

Across all parameter constellations, the overall success rate of scenarios that satisfied all five criteria was 54\%. Put-call parity and negative skew were satisfied in most cases (93\% and 99\%, respectively), while smile convexity was harder to achieve, with a success rate of 66\%. This analysis offers practical guidance for the implementation of the Hypergeometric Volatility Model by identifying parameter regions that consistently produce market-realistic outcomes.

The empirical validation confirmed that the model was able to reproduce essential features of the S\&P~500 volatility surface, particularly in near-the-money and short-maturity regions. However, an interesting discrepancy was observed in the calibration results: the best-fit parameter configuration yielded $H = 0.05$ and $\rho = -0.7$, which fell just outside the optimal ranges identified in the simulation study. Nevertheless, their corresponding success rates remained high and only slightly below those observed in the optimal regions. This shows that parameter regions that perform well in terms of stylized facts may not coincide with those that best fit market data. The calibrated roughness parameter also differed from the theoretical roughness of $H \approx 0.21$ implied by the power-law decay observed in the empirical surface, which highlights the difficulty of linking empirical observations to model parameters. In contrast, the volatility scale $a = 1.0$ and base volatility $b = 0.20$ from the best-fit configuration matched the expected ranges well.

The findings highlight both the strengths and the limitations of the Hypergeometric Volatility Model. On the positive side, the model provides a flexible framework capable of generating rough volatility effects and empirically consistent skew behavior through explicit control of roughness and persistence parameters. The model reproduces key stylized facts when appropriately parameterized and remains computationally feasible through finite-dimensional approximations. Limitations arise in the trade-off between achieving high success across all criteria and obtaining the best fit to market data, as well as in numerical stability at extreme parameter values.

A number of open questions remain that could be addressed in future work. The discrepancy between simulation-optimal and calibration-optimal parameters suggests the need for calibration methods that balance fit quality with stylized fact reproduction, possibly through multi-objective optimization approaches. The current calibration was performed using a grid search approach over the strike-maturity grid, which has higher concentration of data points for shorter maturities and for at-the-money strikes. More advanced calibration methods, such as derivative-free optimization, could be used to navigate the parameter space more efficiently and to search for configurations that reconcile robustness with market fit. In addition, systematic comparisons with alternative rough volatility models could provide a clearer view of the model's relative strengths and weaknesses.

In summary, this thesis has shown that the Hypergeometric Volatility Model is able to reproduce central empirical features of option markets, while also revealing important aspects in the relationship between parameter selection and model performance. The parameter analysis provides useful guidance for practical implementation, but the calibration results emphasize that real-world applications require careful handling of trade-offs between competing modeling objectives. With further methodological refinements and broader empirical testing, the model has the potential to become a valuable tool in both academic research and financial practice.